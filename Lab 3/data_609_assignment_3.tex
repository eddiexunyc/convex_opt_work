% Options for packages loaded elsewhere
\PassOptionsToPackage{unicode}{hyperref}
\PassOptionsToPackage{hyphens}{url}
\PassOptionsToPackage{dvipsnames,svgnames,x11names}{xcolor}
%
\documentclass[
  letterpaper,
  DIV=11,
  numbers=noendperiod]{scrartcl}

\usepackage{amsmath,amssymb}
\usepackage{iftex}
\ifPDFTeX
  \usepackage[T1]{fontenc}
  \usepackage[utf8]{inputenc}
  \usepackage{textcomp} % provide euro and other symbols
\else % if luatex or xetex
  \usepackage{unicode-math}
  \defaultfontfeatures{Scale=MatchLowercase}
  \defaultfontfeatures[\rmfamily]{Ligatures=TeX,Scale=1}
\fi
\usepackage{lmodern}
\ifPDFTeX\else  
    % xetex/luatex font selection
\fi
% Use upquote if available, for straight quotes in verbatim environments
\IfFileExists{upquote.sty}{\usepackage{upquote}}{}
\IfFileExists{microtype.sty}{% use microtype if available
  \usepackage[]{microtype}
  \UseMicrotypeSet[protrusion]{basicmath} % disable protrusion for tt fonts
}{}
\makeatletter
\@ifundefined{KOMAClassName}{% if non-KOMA class
  \IfFileExists{parskip.sty}{%
    \usepackage{parskip}
  }{% else
    \setlength{\parindent}{0pt}
    \setlength{\parskip}{6pt plus 2pt minus 1pt}}
}{% if KOMA class
  \KOMAoptions{parskip=half}}
\makeatother
\usepackage{xcolor}
\setlength{\emergencystretch}{3em} % prevent overfull lines
\setcounter{secnumdepth}{-\maxdimen} % remove section numbering
% Make \paragraph and \subparagraph free-standing
\makeatletter
\ifx\paragraph\undefined\else
  \let\oldparagraph\paragraph
  \renewcommand{\paragraph}{
    \@ifstar
      \xxxParagraphStar
      \xxxParagraphNoStar
  }
  \newcommand{\xxxParagraphStar}[1]{\oldparagraph*{#1}\mbox{}}
  \newcommand{\xxxParagraphNoStar}[1]{\oldparagraph{#1}\mbox{}}
\fi
\ifx\subparagraph\undefined\else
  \let\oldsubparagraph\subparagraph
  \renewcommand{\subparagraph}{
    \@ifstar
      \xxxSubParagraphStar
      \xxxSubParagraphNoStar
  }
  \newcommand{\xxxSubParagraphStar}[1]{\oldsubparagraph*{#1}\mbox{}}
  \newcommand{\xxxSubParagraphNoStar}[1]{\oldsubparagraph{#1}\mbox{}}
\fi
\makeatother

\usepackage{color}
\usepackage{fancyvrb}
\newcommand{\VerbBar}{|}
\newcommand{\VERB}{\Verb[commandchars=\\\{\}]}
\DefineVerbatimEnvironment{Highlighting}{Verbatim}{commandchars=\\\{\}}
% Add ',fontsize=\small' for more characters per line
\usepackage{framed}
\definecolor{shadecolor}{RGB}{241,243,245}
\newenvironment{Shaded}{\begin{snugshade}}{\end{snugshade}}
\newcommand{\AlertTok}[1]{\textcolor[rgb]{0.68,0.00,0.00}{#1}}
\newcommand{\AnnotationTok}[1]{\textcolor[rgb]{0.37,0.37,0.37}{#1}}
\newcommand{\AttributeTok}[1]{\textcolor[rgb]{0.40,0.45,0.13}{#1}}
\newcommand{\BaseNTok}[1]{\textcolor[rgb]{0.68,0.00,0.00}{#1}}
\newcommand{\BuiltInTok}[1]{\textcolor[rgb]{0.00,0.23,0.31}{#1}}
\newcommand{\CharTok}[1]{\textcolor[rgb]{0.13,0.47,0.30}{#1}}
\newcommand{\CommentTok}[1]{\textcolor[rgb]{0.37,0.37,0.37}{#1}}
\newcommand{\CommentVarTok}[1]{\textcolor[rgb]{0.37,0.37,0.37}{\textit{#1}}}
\newcommand{\ConstantTok}[1]{\textcolor[rgb]{0.56,0.35,0.01}{#1}}
\newcommand{\ControlFlowTok}[1]{\textcolor[rgb]{0.00,0.23,0.31}{\textbf{#1}}}
\newcommand{\DataTypeTok}[1]{\textcolor[rgb]{0.68,0.00,0.00}{#1}}
\newcommand{\DecValTok}[1]{\textcolor[rgb]{0.68,0.00,0.00}{#1}}
\newcommand{\DocumentationTok}[1]{\textcolor[rgb]{0.37,0.37,0.37}{\textit{#1}}}
\newcommand{\ErrorTok}[1]{\textcolor[rgb]{0.68,0.00,0.00}{#1}}
\newcommand{\ExtensionTok}[1]{\textcolor[rgb]{0.00,0.23,0.31}{#1}}
\newcommand{\FloatTok}[1]{\textcolor[rgb]{0.68,0.00,0.00}{#1}}
\newcommand{\FunctionTok}[1]{\textcolor[rgb]{0.28,0.35,0.67}{#1}}
\newcommand{\ImportTok}[1]{\textcolor[rgb]{0.00,0.46,0.62}{#1}}
\newcommand{\InformationTok}[1]{\textcolor[rgb]{0.37,0.37,0.37}{#1}}
\newcommand{\KeywordTok}[1]{\textcolor[rgb]{0.00,0.23,0.31}{\textbf{#1}}}
\newcommand{\NormalTok}[1]{\textcolor[rgb]{0.00,0.23,0.31}{#1}}
\newcommand{\OperatorTok}[1]{\textcolor[rgb]{0.37,0.37,0.37}{#1}}
\newcommand{\OtherTok}[1]{\textcolor[rgb]{0.00,0.23,0.31}{#1}}
\newcommand{\PreprocessorTok}[1]{\textcolor[rgb]{0.68,0.00,0.00}{#1}}
\newcommand{\RegionMarkerTok}[1]{\textcolor[rgb]{0.00,0.23,0.31}{#1}}
\newcommand{\SpecialCharTok}[1]{\textcolor[rgb]{0.37,0.37,0.37}{#1}}
\newcommand{\SpecialStringTok}[1]{\textcolor[rgb]{0.13,0.47,0.30}{#1}}
\newcommand{\StringTok}[1]{\textcolor[rgb]{0.13,0.47,0.30}{#1}}
\newcommand{\VariableTok}[1]{\textcolor[rgb]{0.07,0.07,0.07}{#1}}
\newcommand{\VerbatimStringTok}[1]{\textcolor[rgb]{0.13,0.47,0.30}{#1}}
\newcommand{\WarningTok}[1]{\textcolor[rgb]{0.37,0.37,0.37}{\textit{#1}}}

\providecommand{\tightlist}{%
  \setlength{\itemsep}{0pt}\setlength{\parskip}{0pt}}\usepackage{longtable,booktabs,array}
\usepackage{calc} % for calculating minipage widths
% Correct order of tables after \paragraph or \subparagraph
\usepackage{etoolbox}
\makeatletter
\patchcmd\longtable{\par}{\if@noskipsec\mbox{}\fi\par}{}{}
\makeatother
% Allow footnotes in longtable head/foot
\IfFileExists{footnotehyper.sty}{\usepackage{footnotehyper}}{\usepackage{footnote}}
\makesavenoteenv{longtable}
\usepackage{graphicx}
\makeatletter
\newsavebox\pandoc@box
\newcommand*\pandocbounded[1]{% scales image to fit in text height/width
  \sbox\pandoc@box{#1}%
  \Gscale@div\@tempa{\textheight}{\dimexpr\ht\pandoc@box+\dp\pandoc@box\relax}%
  \Gscale@div\@tempb{\linewidth}{\wd\pandoc@box}%
  \ifdim\@tempb\p@<\@tempa\p@\let\@tempa\@tempb\fi% select the smaller of both
  \ifdim\@tempa\p@<\p@\scalebox{\@tempa}{\usebox\pandoc@box}%
  \else\usebox{\pandoc@box}%
  \fi%
}
% Set default figure placement to htbp
\def\fps@figure{htbp}
\makeatother

\KOMAoption{captions}{tableheading}
\makeatletter
\@ifpackageloaded{caption}{}{\usepackage{caption}}
\AtBeginDocument{%
\ifdefined\contentsname
  \renewcommand*\contentsname{Table of contents}
\else
  \newcommand\contentsname{Table of contents}
\fi
\ifdefined\listfigurename
  \renewcommand*\listfigurename{List of Figures}
\else
  \newcommand\listfigurename{List of Figures}
\fi
\ifdefined\listtablename
  \renewcommand*\listtablename{List of Tables}
\else
  \newcommand\listtablename{List of Tables}
\fi
\ifdefined\figurename
  \renewcommand*\figurename{Figure}
\else
  \newcommand\figurename{Figure}
\fi
\ifdefined\tablename
  \renewcommand*\tablename{Table}
\else
  \newcommand\tablename{Table}
\fi
}
\@ifpackageloaded{float}{}{\usepackage{float}}
\floatstyle{ruled}
\@ifundefined{c@chapter}{\newfloat{codelisting}{h}{lop}}{\newfloat{codelisting}{h}{lop}[chapter]}
\floatname{codelisting}{Listing}
\newcommand*\listoflistings{\listof{codelisting}{List of Listings}}
\makeatother
\makeatletter
\makeatother
\makeatletter
\@ifpackageloaded{caption}{}{\usepackage{caption}}
\@ifpackageloaded{subcaption}{}{\usepackage{subcaption}}
\makeatother

\usepackage{bookmark}

\IfFileExists{xurl.sty}{\usepackage{xurl}}{} % add URL line breaks if available
\urlstyle{same} % disable monospaced font for URLs
\hypersetup{
  pdftitle={DATA 609 - Homework 3: Convex Sets},
  pdfauthor={Eddie Xu},
  colorlinks=true,
  linkcolor={blue},
  filecolor={Maroon},
  citecolor={Blue},
  urlcolor={Blue},
  pdfcreator={LaTeX via pandoc}}


\title{DATA 609 - Homework 3: Convex Sets}
\author{Eddie Xu}
\date{}

\begin{document}
\maketitle


\subsection{Instructions}\label{instructions}

Please submit a .qmd file along with a rendered pdf to the Brightspace
page for this assignment. You may use whatever language you like within
your qmd file, I recommend python, julia, or R.

\subsection{Problem 1 (cvx-book 2.12):}\label{problem-1-cvx-book-2.12}

Which of the following sets are convex? For each case give the reason(s)
why or why not

\begin{enumerate}
\def\labelenumi{\alph{enumi}.}
\tightlist
\item
  A slab, i.e., a set of the form
  \(\{x \in \mathbb{R}^n\, |\, \alpha \leq \mathbf{a}^T \mathbf{x} \leq{\beta}\}\).
\item
  A rectangle, i.e., a set of the form
  \(\{x \in \mathbb{R}^n\, |\, \alpha_i \leq x_i \leq \beta_i, i = 1,\cdots,\, n\}\).
  A rectangle is sometimes called a hyperrectangle when n \textgreater{}
  2.
\item
  A wedge, i.e.,
  \(\{ \mathbf{x} \in \mathbf{R}^n\, |\, \mathbf{a_1}^T\mathbf{x} \leq \mathbf{b}_1, \mathbf{a}_2^T\leq\mathbf{b}_2\}\)
\item
  The set of points closer to a given point than a given set, i.e.,
  \(\{ \mathbf{x}\, |\, \|\mathbf{x} − \mathbf{x}_0\|^2 \leq \|\mathbf{x} − \mathbf{y}\|^2\)
  for all \(y \in S\}\) where \(S \subset \mathbb{R}^n\).
\item
  The set of points closer to one set than another, i.e.,
  \(\{\mathbf{x}\ | dist(\mathbf{x}, S) \leq dist(\mathbf{x}, T )\}\),
  where \(S\), \(T\) are subsets of \(\mathbf{R}^N\), and
  \(dist(x, S) = \inf\{\|\mathbf{x}  − \mathbf{z}\|^2 | \mathbf{z} \in S\}\).
\end{enumerate}

Problem 1 Solution

\begin{enumerate}
\def\labelenumi{\alph{enumi}.}
\tightlist
\item
  The slab is a convex set because it is an intersection of two
  halfspaces.
\item
  The rectangle is a convex set because the set contains a finite
  intersection of halfspaces.
\item
  The wedge set is a convex set because like part a, it is an
  intersection of two halfspaces.
\item
  The set of points is a convex set because the set can be expressed as
  an intersection of halfspaces where
  \(\quad S = \{ x \mid \| x - x_0 \|_2 \leq \| x - y \|_2 \}\) for
  fixed y
\item
  The set of points is not a convex set because for
  \(\quad {x | dist(x, S) ≤ dist(x, T)}\), both subset \(S\) and \(T\)
  are not convex sets when \(\quad{x ∈ R | x ≤ −1/2}\) or
  \(\quad {x ≥ 1/2}\)
\end{enumerate}

\subsection{Problem 2 (cvx-book 2.15):}\label{problem-2-cvx-book-2.15}

Some sets of probability distributions. Let \(x\) be a real-valued
random variable with probability distribution

\[\mathbf{prob}(x = a_i) = p_i, i = 1, \cdots, n\]

where: \(a_1 < a_2 < \cdots < a_n\). Of course
\(\mathbf{p} \in \mathbb{R}^n\) lies in the standard probability simplex
\(P \{\mathbf{p}\, |\, 1^T \mathbf{p} = 1, \mathbf{p} \preceq 0\}\).

Which of the following conditions are convex in \(\mathbf{p}\)? (That
is, for which of the following conditions is the set of
\(\mathbf{p} \in P\) that satisfy the condition convex?) For each case
give the reason(s) why or why not.

Which of the following conditions are convex in \(\mathbf{p}\)? (That
is, for which of the following conditions is the set of
\(\mathbf{p} \in P\) that satisfy the condition convex?) For each case
give the reason(s) why or why not.

\begin{enumerate}
\def\labelenumi{\alph{enumi}.}
\tightlist
\item
  The set of all \(\mathbf{p}\) where the expectation of the function
  \(f(x)\) is between two limits: \(\alpha \leq Ef(x) \leq \beta\),
  \(Ef(x) = \sum_{i=1}^n p_i f(a_i)\). Here \(f(x)\) is a function from
  \(\mathbb{R}\) to \(\mathbb{R}\).
\item
  The set of all \(\mathbf{p}\) such that the probability that
  \(\mathbf{prob}(x\leq\alpha) \leq \beta\)
\item
  The set of all \(\mathbf{p}\) such that the expectation of \(|x|^3\)
  is greater than a given constant \(\alpha\) times the expectation of
  \(|x|\): \$E\textbar x\^{}3\textbar{} \(\leq\) \(\α E|x|\)
\item
  The set of all \(\mathbf{p}\) such that the expectation of \(x^2\) is
  less than a given constant \(\alpha\): \(Ex^2 \leq \alpha\)
\end{enumerate}

Problem 2 Solution

\begin{enumerate}
\def\labelenumi{\alph{enumi}.}
\tightlist
\item
  \(Ef(x) = \sum_{i=1}^n p_i f(a_i)\) can be defined as a convex set
  since the constraint is equivalent to two linear inequalities in the
  probabilities \(p_i\).
\item
  The set of all \(\mathbf{p}\) such that the probability that
  \(\mathbf{prob}(x\leq\alpha) \leq \beta\) can be defined as a convex
  set because the constraint is equivalent to a linear inequality in the
  probabilities \(p_i\).
\item
  The set of all \(\mathbf{p}\) such that the expectation of \(|x|^3\)
  is greater than a given constant \(\alpha\) times the expectation of
  \(|x|\): \$E\textbar x\^{}3\textbar{} \(\leq\) \(\α E|x|\) can be
  defined as a convex set because the constraint is equivalent to a
  linear inequality in the probabilities \(p_i\).
\item
  The set of all \(\mathbf{p}\) such that the expectation of \(x^2\) is
  less than a given constant \(\alpha\): \(Ex^2 \leq \alpha\) can be
  defined as a convex set because the constraint is equivalent to a
  linear inequality in the probabilities \(p_i\).
\end{enumerate}

\subsection{Problem 3: Bounded Value Least Squares for Wine
Mixing}\label{problem-3-bounded-value-least-squares-for-wine-mixing}

We have seen several examples so far in the couse where we would like to
have inequality constraints on the decision variable for our least
squares problem, for example to prevent non-sensical solutions like
spending a negative amount of money on advertising, limiting the total
investment in certain types of assets, or perhaps bounding the value of
a statistical coefficient to a certain range. Non-negative least squares
is a type of least squares problem where the decision variables
\(\mathbf{x}\geq 0\), and Bounded-Value Least Squares allows for more
general constraints.

This type of least-squares problem needs to be solved algorithmically,
and we will use it to get our first practice using the \texttt{CVX}
software package. You should install \texttt{CVXPY}, \texttt{CVXR}, or a
flavor of \texttt{CVX} compatible with whatever software you are using
to solve the problem and use \texttt{CVX} to solve this problem.

The problem is one of finding a mixture of wines which achieves certain
chemical characteristics. I have attached a dataset which contains data
on the chemical composition of 6 different wines (the dataset originates
from kaggle but is reduced for our purposes). Each wine is described
according to 11 chemical characteristics, including \texttt{alcohol},
\texttt{residual\ sugar}, \texttt{chlorides}, etc. I have also provided
data for the chemical composition for a target wine.

\begin{itemize}
\tightlist
\item
  \href{https://github.com/georgehagstrom/DATA609Spring2025/blob/main/website/assignments/labs/labData/wine_data.csv}{wine\_data.csv}
\item
  \href{https://github.com/georgehagstrom/DATA609Spring2025/blob/main/website/assignments/labs/labData/target.csv}{target.csv}
\end{itemize}

The goal of this problem is to find the blend of wines which has
chemical characteristics closest to the target wine.

Concretely, you are solving for weights \(\mathbf{p}\). The
concentration of chemical \(i\) in wine \(j\) is given by the matrix
\(C_{ij}\), and the concentration in the blended wine is:

\[
c_{blend,i} = \sum_{j=1}^6 C_{ij} p_j,
\] so that the overall concentration vector in the blend satisfies: \[
\mathbf{c}_{blend} = C\mathbf{p}
\]

The vector \(\mathbf{p}\) is a discrete probability distribution,
meaning that all entries are non-negative and must sum to \(1\) (you
can't add negative wine). The range of each chemical varies greatly, so
our objective function should incorporate a penalty that is weighted
according to the magnitude of the value in the target function:

\[
\min_{\mathbf{p}} \sum_{i=1}^{11} \left(\frac{c_i-c_{blend,_i}}{c_i}\right)^2
\]

Implement this least squares optimization problem using CVX and
determine the optimal blend of wines to match the target.

\begin{Shaded}
\begin{Highlighting}[]
\CommentTok{\# load packages}
\ImportTok{import}\NormalTok{ numpy }\ImportTok{as}\NormalTok{ np}
\ImportTok{import}\NormalTok{ pandas }\ImportTok{as}\NormalTok{ pd}
\ImportTok{import}\NormalTok{ cvxpy }\ImportTok{as}\NormalTok{ cp}

\CommentTok{\# extract data}
\NormalTok{wine\_url }\OperatorTok{=} \StringTok{\textquotesingle{}https://media.githubusercontent.com/media/georgehagstrom/DATA609Spring2025/refs/heads/main/website/assignments/labs/labData/wine\_data.csv\textquotesingle{}}
\NormalTok{target\_url }\OperatorTok{=} \StringTok{\textquotesingle{}https://media.githubusercontent.com/media/georgehagstrom/DATA609Spring2025/refs/heads/main/website/assignments/labs/labData/target.csv\textquotesingle{}}

\NormalTok{wine\_data }\OperatorTok{=}\NormalTok{ pd.read\_csv(wine\_url)}
\NormalTok{target\_data }\OperatorTok{=}\NormalTok{ pd.read\_csv(target\_url)}

\CommentTok{\# extract the relevant data and define the decision variables}
\NormalTok{C }\OperatorTok{=}\NormalTok{ wine\_data.values}
\NormalTok{target }\OperatorTok{=}\NormalTok{ target\_data.values.flatten() }
\NormalTok{p }\OperatorTok{=}\NormalTok{ cp.Variable(}\DecValTok{6}\NormalTok{, nonneg}\OperatorTok{=}\VariableTok{True}\NormalTok{)}

\CommentTok{\# define the objective function}
\NormalTok{objective }\OperatorTok{=}\NormalTok{ cp.Minimize(cp.norm(np.diag(C) }\OperatorTok{@}\NormalTok{ p }\OperatorTok{{-}}\NormalTok{ target))}

\CommentTok{\# solve the problem}
\NormalTok{problem }\OperatorTok{=}\NormalTok{ cp.Problem(objective)}
\NormalTok{problem.solve()}

\CommentTok{\# get the optimal weights for the wines}
\NormalTok{optimal\_weights }\OperatorTok{=}\NormalTok{ p.value}
\BuiltInTok{print}\NormalTok{(}\SpecialStringTok{f"Optimal wine blend weights: }\SpecialCharTok{\{}\NormalTok{optimal\_weights}\SpecialCharTok{\}}\SpecialStringTok{"}\NormalTok{)}
\end{Highlighting}
\end{Shaded}

\begin{verbatim}
Optimal wine blend weights: [0.02638134 3.18943341 4.38723391 1.20138733 5.65966594 0.04696901]
\end{verbatim}




\end{document}
